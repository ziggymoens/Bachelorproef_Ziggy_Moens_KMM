%%=============================================================================
%% Inleiding
%%=============================================================================

\chapter{\IfLanguageName{dutch}{Inleiding}{Introduction}}
\label{ch:inleiding}

Cross-platform ontwikkeling is een gegeven dat in de laatste jaren aan populariteit wint. Dit vooral omdat er een paar voordelen aan verbonden zijn die voor bedrijven gespecialiseerd in software voor verschillende platformen zeer interessant kunnen zijn. Hierbij treden vooral de snellere ontwikkelingstijd en de lagere ontwikkelingskosten naar de voorgrond. Deze twee zaken spelen dus voor die bedrijven een grote rol binnen hun dag dagelijkse werking. Losstaand van deze voordelen is nog niet elk bedrijf te vinden voor cross-platform ontwikkeling. Het beeld dat deze cross-platform applicaties `beter' zouden zijn dan de native oplossingen op de markt is nog niet overal aanvaard. Kotlin Multiplatform Mobile (KMM) is de nieuwe cross-platform software development kit van JetBrains zou hier verandering in kunnen brengen. KMM gaat voor een andere aanpak voor hun cross-platform applicaties en zal hier vooral gaan focussen op een gedeelde business logica en een platformafhankelijke en dus native user interface. 

%De inleiding moet de lezer net genoeg informatie verschaffen om het onderwerp te begrijpen en in te zien waarom de onderzoeksvraag de moeite waard is om te onderzoeken. In de inleiding ga je literatuurverwijzingen beperken, zodat de tekst vlot leesbaar blijft. Je kan de inleiding verder onderverdelen in secties als dit de tekst verduidelijkt. Zaken die aan bod kunnen komen in de inleiding~\autocite{Pollefliet2011}:

%\begin{itemize}
%  \item context, achtergrond
%  \item afbakenen van het onderwerp
%  \item verantwoording van het onderwerp, methodologie
%  \item probleemstelling
%  \item onderzoeksdoelstelling
%  \item onderzoeksvraag
%  \item \ldots
%\end{itemize}

\section{\IfLanguageName{dutch}{Probleemstelling}{Problem Statement}}
\label{sec:probleemstelling}

Veel bedrijven die zich specialiseren in applicatieontwikkeling voor mobile toestellen zoals Android en iOS toestellen hebben verschillende mogelijkheden om dit aan te pakken. Deze kunnen bijvoorbeeld kiezen voor een native applicatie of een cross-platform-applicatie. Anderzijds zijn er nog mogelijkheden zoals een progressive web application (PWA) geschreven in React native of Angular. De PWA's vallen echter buiten de scope van deze bachelorproef. Er zal dus vooral gefocust worden op bedrijven die momenteel native applicaties ontwikkelen en eventueel willen overschakelen naar cross-platform mochten zij hier grote voordelen uithalen.


%Uit je probleemstelling moet duidelijk zijn dat je onderzoek een meerwaarde heeft voor een concrete doelgroep. De doelgroep moet goed gedefinieerd en afgelijnd zijn. Doelgroepen als ``bedrijven,'' ``KMO's,'' systeembeheerders, enz.~zijn nog te vaag. Als je een lijstje kan maken van de personen/organisaties die een meerwaarde zullen vinden in deze bachelorproef (dit is eigenlijk je steekproefkader), dan is dat een indicatie dat de doelgroep goed gedefinieerd is. Dit kan een enkel bedrijf zijn of zelfs één persoon (je co-promotor/opdrachtgever).

\section{\IfLanguageName{dutch}{Afbakening van het onderwerp}{Demarcation of the subject}}
\label{sec:afbakening}

JetBrains heeft Kotlin Multiplatform (KM) uitgebracht met Kotlin Multiplatform Mobile (KMM) als specialisatie voor de mobile toestellen. KMM is  vooral gericht op iPhone en Android in tegenstelling tot KM dat zich ook zal gaan toespitsen op Mac, Windows ... Voor deze bachelorproef zal er dus vooral gefocust worden op KMM aangezien de onderzoeksvraag zich toespitst op de mobile toestellen. KMM biedt ook dezelfde functies die KM zal aanbieden aan de ontwikkelaars aangezien KMM een aftakking is van KM. Voor besturingssystemen van mobile toestellen zal er vooral gekeken worden naar Android en iOS aangezien deze de meest gebruikte besturingssystemen zijn op de markt.


\section{\IfLanguageName{dutch}{Onderzoeksvraag}{Research question}}
\label{sec:onderzoeksvraag}

De opzet van deze bachelorproef is na te gaan of native applicatieontwikkeling voor Android en iOS apparaten nog steeds de beste keuze is of KMM een sneller en efficiënter alternatief kan vormen binnen het huidige beeld van applicatieontwikkeling. Om dit te onderzoeken zijn er verschillende testcriteria opgesteld. Deze testcriteria zijn: 
\begin{itemize}
    \item het aantal lijnen code
    \item de kostprijs
    \item de ontwikkeltijd
    \item de compileersnelheid
    \item de voetafdruk
    \item de uitbreidingsmogelijkheden van de applicaties
\end{itemize}
Aan de hand hiervan kan besloten worden of de KMM-applicaties een beter alternatief kunnen vormen voor native applicaties. Daarnaast zal ook de vraag gesteld worden of het huidige beeld van de SWOT-analyse voor cross-platform applicaties ook toepasbaar is voor KMM-applicaties, indien niet zal gekeken worden hoe deze dient aangepast te worden. 

%Wees zo concreet mogelijk bij het formuleren van je onderzoeksvraag. Een onderzoeksvraag is trouwens iets waar nog niemand op dit moment een antwoord heeft (voor zover je kan nagaan). Het opzoeken van bestaande informatie (bv. ``welke tools bestaan er voor deze toepassing?'') is dus geen onderzoeksvraag. Je kan de onderzoeksvraag verder specifiëren in deelvragen. Bv.~als je onderzoek gaat over performantiemetingen, dan 

\section{\IfLanguageName{dutch}{Onderzoeksdoelstelling}{Research objective}}
\label{sec:onderzoeksdoelstelling}

Voor deze bachelor zal bekeken worden of Kotlin Multiplatofrm Mobile (KMM) een goed aternatief kan bieden voor native ontwikkeling. Hiervoor zullen de een vergelijkende studie uitvoeren tussen KMM en native applicaties. Voor deze studie zullen meerdere applicaties geschreven worden voor zowel Android als iOS. Eerst zullen er 2 native applicaties geschreven worden. De native iOS applicatie zal geschreven worden in Swift 5.3 of recenter met iOS 14 in gedachten en voor de user interface zal er gebruik gemaakt worden van UIKit. Aan de Android kant van het native verhaal zal er een applicatie geschreven worden met behulp van Kotlin 1.4.20 of een recentere versie. Voor de native Android user interface zal gebruik gemaakt worden van de standaard Android en AndroidX bibliotheek. Naast de native applicaties zal er nog een derde applicatie geschreven worden, namelijk een Kotlin Multiplatform Mobile (KMM) applicatie. Deze applicatie zal werken op zowel Android als iOS toestellen. Hierbij zal voor het iOS-deel gebruik gemaakt worden van Swift en SwiftUI voor de user interface en voor het Android deel van Kotlin en Jetpack Compose.
\\ \\
Om deze manier van werken correct te beoordelen is het belangrijk dat zowel de native applicaties als KMM dezelfde functionaliteiten zullen aanbieden. Hierbij geldt ook voor de native applicaties voor Android en iOS onderling. De geschreven applicaties zouden dus als het ware niet van elkaar te mogen onderscheiden zijn. Hierbij dient echter rekening gehouden te worden met het feit dat de user interface wel enigszins kan verschillen.
\\ \\
Voor de applicaties zelf zal er gekeken worden naar de voorbeeld applicaties die JetBrains aanbiedt en voorstelt. Dit zijn onder andere de PeopleInSpace applicatie \autocite{OReilly2021} en de SpaceX lauches applicatie \autocite{Kotlin2020HandsOn}. Deze kunnen gebruikt worden om benchmarks op te testen en als inspiratiebron voor de native applicaties en andere KMM-applicaties.
\\ \\
Om de uiteindelijke de vergelijkende studie te maken tussen de native applicaties en de KMM-applicatie zal er gekeken worden naar verschillende factoren. Deze factoren zijn
\begin{itemize}
    \item Aantal lijnen code
    \begin{itemize}
        \item Het totaal aantal lijnen code van een applicatie zal geëvalueerd worden over het gehele project. In het geval van native applicaties zullen deze opgeteld worden met elkaar.
    \end{itemize}
    \item Kostprijs
    \begin{itemize}
        \item Hierbij wordt de geschatte kostprijs om een applicatie te laten ontwikkelen door een IT-bedrijf in kaart gebracht. Dit wordt berekend aan de hand van geschatte werkuren en een gemiddelde kostprijs per uur. Daarnaast wordt nagegaan of cross-platform een effect zal hebben op het systeem van vooraf bepaalde totaalprijzen indien bedrijven daarmee werken.
    \end{itemize}
    \item Ontwikkeltijd
    \begin{itemize}
        \item Deze tijd beschrijft het aantal werkuren dat een ontwikkelaar nodig heeft om een specifieke applicatie te schrijven. Hierbij kan onder andere gebruik gemaakt worden van platformen zoals GitHub om deze tijd te gaan meten of inschatten.
    \end{itemize}
    \item Compileersnelheid
    \begin{itemize}
        \item Dit is de snelheid waarmee de specifieke applicatie zal kunnen compileren en opstarten. Dit kan gemeten worden in de ontwikkelingssoftware voor de desbetreffende taal van de applicatie.
    \end{itemize}
    \item Voetafdruk
    \begin{itemize}
        \item Dit impliceert de omvang die de applicatie zal innemen op het platform waarvoor deze ontwikkeld is. Hiervoor kan de applicatie gebruikt worden die de ontwikkelingssoftware aanmaakt.
    \end{itemize}
    \item Uitbreiding van de applicatie
    \begin{itemize}
        \item Dit criterium kan geëvalueerd worden door vooraf bepaalde features van de applicatie weg te laten. Eens de applicatie klaar is voor productie kunnen deze features terug toegevoegd worden. Om de uitbreidbaarheid van de applicatie te gaan staven kan gebruik gemaakt worden van voorgaande testcriteria.
    \end{itemize}
\end{itemize}

Voor dit onderzoek zullen volgende hardware en software gebruikt worden:\\
Hardware:
\begin{itemize}
    \item MacBook Pro 16 inch uit 2019
    \item iPhone 11 Pro Max uit 2019
    \item Huawei P9 lite uit 2016
\end{itemize}
Deze laatste twee toestellen zullen echter minder van belang zijn en zullen enkel ter sprake komen indien er testen gebeuren op fysieke toestellen. Voor het grotere deel van de testen zullen emulators gebruikt worden die ingebouwd zijn in de gekozen ontwikkelingssoftware.\\
\\
Software:
\begin{itemize}
    \item Xcode versie 12.3 of recenter
    \item Android studio versie 4.1.1 of recenter
\end{itemize}
Voor de KMM-applicaties zal er echter nog de Kotlin Multiplatform Mobile plugin geïnstalleerd moeten worden.

Deze software zal gebruikt worden op de MacBook Pro die hierboven reeds werd beschreven. Voor dit onderzoek geldt de beperking dat er enkel gebruik gemaakt kan worden van toestellen die MacOS gebruiken als besturingssysteem, aangezien dit een vereiste is voor de ontwikkeling van iOS-applicaties.
\\ \\ 
KMM is een zeer recente technologie die volop in ontwikkeling is. Daardoor wordt verwacht dat de voor deze studie resultaten nog niet het volle potentieel zullen aantonen. Dit wil echter niet zeggen dat de technologie geen meerwaarde kan bieden voor bedrijven die zich momenteel bezighouden met native applicatieontwikkeling voor zowel Android als iOS. Enkele componenten van de SDK staan momenteel nog niet op punt, dit zal waarschijnlijk nog verbeteren naar de toekomst toe. Het onderzoek kan, gezien de prematuriteit van KMM, een tijdelijke referentie bieden omtrent KMM en de mogelijkheid dit als een sneller en efficiënter alternatief te gebruiken voor native applicaties. Dit is vooral interessant voor bedrijven die zich momenteel vooral toespitsen op native ontwikkeling en eventueel naar de toekomst toe de overstap willen maken naar KMM. 


%Wat is het beoogde resultaat van je bachelorproef? Wat zijn de criteria voor succes? Beschrijf die zo concreet mogelijk. Gaat het bv. om een proof-of-concept, een prototype, een verslag met aanbevelingen, een vergelijkende studie, enz.

\section{\IfLanguageName{dutch}{Opzet van deze bachelorproef}{Structure of this bachelor thesis}}
\label{sec:opzet-bachelorproef}

% Het is gebruikelijk aan het einde van de inleiding een overzicht te
% geven van de opbouw van de rest van de tekst. Deze sectie bevat al een aanzet
% die je kan aanvullen/aanpassen in functie van je eigen tekst.

Het vervolg van de bachelorproef wordt als volgt opgebouwd:

In Hoofdstuk~\ref{ch:stand-van-zaken} wordt een overzicht gegeven van de stand van zaken binnen het onderzoeksdomein, op basis van een literatuurstudie.

In Hoofdstuk~\ref{ch:methodologie} wordt de methodologie toegelicht en worden de gebruikte onderzoekstechnieken besproken om een antwoord te kunnen formuleren op de onderzoeksvragen.

% TODO: Vul hier aan voor je eigen hoofstukken, één of twee zinnen per hoofdstuk

In Hoofdstuk~\ref{ch:conclusie}, tenslotte, wordt de conclusie gegeven en een antwoord geformuleerd op de onderzoeksvragen. Daarbij wordt ook een aanzet gegeven voor toekomstig onderzoek binnen dit domein.