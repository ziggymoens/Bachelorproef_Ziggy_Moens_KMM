%%=============================================================================
%% Samenvatting
%%=============================================================================

% TODO: De "abstract" of samenvatting is een kernachtige (~ 1 blz. voor een
% thesis) synthese van het document.
%
% Deze aspecten moeten zeker aan bod komen:
% - Context: waarom is dit werk belangrijk?
% - Nood: waarom moest dit onderzocht worden?
% - Taak: wat heb je precies gedaan?
% - Object: wat staat in dit document geschreven?
% - Resultaat: wat was het resultaat?
% - Conclusie: wat is/zijn de belangrijkste conclusie(s)?
% - Perspectief: blijven er nog vragen open die in de toekomst nog kunnen
%    onderzocht worden? Wat is een mogelijk vervolg voor jouw onderzoek?
%
% LET OP! Een samenvatting is GEEN voorwoord!

%%---------- Nederlandse samenvatting -----------------------------------------
%
% TODO: Als je je bachelorproef in het Engels schrijft, moet je eerst een
% Nederlandse samenvatting invoegen. Haal daarvoor onderstaande code uit
% commentaar.
% Wie zijn bachelorproef in het Nederlands schrijft, kan dit negeren, de inhoud
% wordt niet in het document ingevoegd.

\IfLanguageName{english}{%
\selectlanguage{dutch}
\chapter*{Samenvatting}
\selectlanguage{english}
}{}

%%---------- Samenvatting -----------------------------------------------------
% De samenvatting in de hoofdtaal van het document

\chapter*{\IfLanguageName{dutch}{Samenvatting}{Abstract}}

Binnen deze bachelorproef wordt nagegaan of Kotlin Multiplatform Mobile een alternatief kan bieden voor native applicatieontwikkeling. Cross-platform ontwikkeling lijkt de laatste tijd steeds interessanter gezien de doorgaans lagere kosten en snellere ontwikkeltijden. In deze studie worden  gebruikt gemaakt van volgende applicaties: 
\begin{itemize}
    \item Native Android applicatie
    \item Native iOS applicatie
    \item Kotlin Multiplatform Mobile applicatie met ondersteuning voor Android en iOS
\end{itemize}
Deze applicaties waren vooral gefocust op het vergelijken van de core van de applicaties, hierdoor hadden deze echter niet veel inhoud. Met deze applicaties werden de verschillen en gelijkenissen opgemeten tussen de native en cross-platform applicaties. Hiervoor werden volgende testcriteria gebruikt: het aantal lijnen code, kostprijs, ontwikkeltijd, compileertijd en de voetafdruk.
\\ \\
Uit deze resultaten is gebleken dat Kotlin Multiplatform Mobile veel potentieel toont als alternatief voor native applicaties. Bij het evalueren van deze resultaten dient wel het feit dat Kotlin Multiplatform Mobile zich nog altijd in het alfa stadium bevindt, in het achterhoofd gehouden te worden. Volgende testen speelden in het voordeel van Kotlin Multiplatform Mobile: het aantal lijnen code en de eerste compilatie. Daarnaast waren de native applicaties iets performanter op vlak van voetafdruk.
\\ \\ 
De testcriteria die de ontwikkelingstijd en de kostprijs evalueerden toonden een klein voordeel voor Kotlin Multiplatform Mobile. Aangezien Kotlin Multiplatform Mobile net sneller te ontwikkelen was kwam deze dus ook als goedkoopste naar boven. Eens de projectmanagementkosten werden ingecalculeerd, was het verschil groter in het voordeel van Kotlin Multiplatform Mobile.
\\ \\
Uiteindelijk kenden de cross-platform applicaties dezelfde look en feel van de native applicaties en hadden ongeveer gelijke of zelfs betere resultaten op vlak van snelheid, performantie en ontwikkelingstijd. Dit maakt dat Kotlin Multiplatform Mobile een interessant alternatief kan zijn voor native applicatieontwikkeling. Eens Kotlin Multiplatform Mobile in productie gaat en bedrijven maken de overstap wordt verondersteld dat Kotlin Multiplatform Mobile hier zeker een voordeel kan bieden.
\\ \\
