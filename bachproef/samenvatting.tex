%%=============================================================================
%% Samenvatting
%%=============================================================================

% TODO: De "abstract" of samenvatting is een kernachtige (~ 1 blz. voor een
% thesis) synthese van het document.
%
% Deze aspecten moeten zeker aan bod komen:
% - Context: waarom is dit werk belangrijk?
% - Nood: waarom moest dit onderzocht worden?
% - Taak: wat heb je precies gedaan?
% - Object: wat staat in dit document geschreven?
% - Resultaat: wat was het resultaat?
% - Conclusie: wat is/zijn de belangrijkste conclusie(s)?
% - Perspectief: blijven er nog vragen open die in de toekomst nog kunnen
%    onderzocht worden? Wat is een mogelijk vervolg voor jouw onderzoek?
%
% LET OP! Een samenvatting is GEEN voorwoord!

%%---------- Nederlandse samenvatting -----------------------------------------
%
% TODO: Als je je bachelorproef in het Engels schrijft, moet je eerst een
% Nederlandse samenvatting invoegen. Haal daarvoor onderstaande code uit
% commentaar.
% Wie zijn bachelorproef in het Nederlands schrijft, kan dit negeren, de inhoud
% wordt niet in het document ingevoegd.

\IfLanguageName{english}{%
\selectlanguage{dutch}
\chapter*{Samenvatting}
\selectlanguage{english}
}{}

%%---------- Samenvatting -----------------------------------------------------
% De samenvatting in de hoofdtaal van het document

\chapter*{\IfLanguageName{dutch}{Samenvatting}{Abstract}}

Binnen deze bachelorproef werd nagegaan of Kotlin Multiplatform Mobile een alternatief kan bieden voor native applicatieontwikkeling. Cross-platform ontwikkeling wordt de laatste tijd steeds interessanter gezien de doorgaans lagere kosten en snellere ontwikkeltijden. Binnen deze studie werd aan de hand van twee native applicaties voor Android en iOS geschreven en daarnaast ook een Kotlin Multiplatform Mobile applicatie die zowel Android als iOS ondersteunt. Met deze applicaties werden de verschillen en gelijkenissen opgemeten tussen de native en cross-platform applicaties. Deze applicaties zijn vooral gefocust op het vergelijken van de core van de applicaties, hierdoor bevatten deze echter niet veel inhoud. 
\\ \\
Uit deze resultaten is gebleken dat Kotlin Multiplatform Mobile veel potentieel toont als alternatief voor native applicaties. Bij het evalueren van deze resultaten dient wel het feit dat Kotlin Multiplatform Mobile nog in het alpha stadium is, in achterhoofd gehouden te worden. De cross-platform applicaties bezaten dezelfde look en feel van de native applicaties en hadden gelijke of zelfs betere resultaten op vlak van snelheid, performantie en ontwikkelingstijd. 
\\ \\
